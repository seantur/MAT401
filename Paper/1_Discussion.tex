The quaternion equation was briefly introduced in Equation \ref{eq:quat}.
The following definition is a rigorous definion that follows from that equation.
\begin{defn}[Quaternion]
A \textit{quaternion} is a number of the form $$a + b\qi + c\qj + c\qk $$ where a, b, c, and d are real numbers, \qi, \qj, \qk, are square roots of -1, and \qi\qj\qk = -1.
\end{defn}
\noindent The following section will describe quaternions in more depth, specifically related to algebra, geometry, and differential calculus.
\subsection{Algebra \& Quaternions}
The addition and subtraction of quaternions is the same as 4D vector addition.
That is, adding quaternions is simply separately adding the coefficients of \qi, \qj, \qk.
For example, $$ (1 + 2\qi + 3\qj + 4\qk) + (-2 + 3\qi - 1\qj + 4\qk) = -1 + 5\qi + 2\qj + 8\qk$$
\noindent Multiplication is a little more involved.
Clearly, since \qi, \qj, and \qk  are square roots of -1, it is true that $\qi^2 = \qj^2 = \qk^2 = -1$.
However, it is not so clear what, for example, $\qi\qj$ is.
We know that $\qi\qj \neq -1$ because $\qi \qj \qk = -1$, and $\qk \neq 1$.
To find $\qi\qj$ we must use Equation \ref{eq:quat} as shown: $$ \qi\qj = -\qi\qj(-1) = -\qi\qj\qk^2 = -(\qi\qj\qk)\qk = -(-1)\qk = \qk$$
It is important to note that quaternion multiplication is not commutative, since $ \qi\qj = \qk \neq \qj\qi = -\qk $.
A full table of the quaternion relationships between \qi, \qj, and \qk$ $ are shown below in Table \ref{tab:quat}:
\begin{table}[H]
\centering
\caption{Quaternion Characteristics}
\label{tab:quat}
\begin{tabular}{|l|l|l|l|}
\hline
 & \qi & \qj & \qk \\ \hline
\qi & \text{\textbf{-1}} & \qk & -\qj \\ \hline
\qj & -\qk & \text{\textbf{-1}} & \qi \\ \hline
\qk & -\qj & \qi & \text{\textbf{-1}} \\ \hline
\end{tabular}
\end{table}

Quaternion multiplication is not commutative as shown above, but other algebraic properties are satisfied as shown in Theorem \ref{thm:mult}

\begin{thm}
\label{thm:mult}
\begin{enumerate} \textit{Properties of Quaternion Multiplication}
	\item Associativity: $\qq ( \textbf{r} \textbf{s}) = (\textbf{q} \textbf{r}) \textbf{s}$
	\item Distributivity: $\qq (\textbf{r} + \textbf{s}) = \textbf{q} \textbf{r} + \textbf{q} \textbf{s}$
	\item Inverses: $\forall$ quaternions $\qq \neq 0$, $\exists$ a quaternion $\textbf{r}$ s.t. $\textbf{qr} = 1$
	\item Cancellation: If $\textbf{qr}=\textbf{qs}$, then $\textbf{r} = \textbf{s}$
\end{enumerate}

\end{thm}

\noindent When quaternions are written in the form $\qq = a + b\qi + c\qj + d\qk$, they are said to be in \textit{Cartesian form}, similar to the method of displaying a complex number in the form $a + bi$.
Just as we can separate a complex number into real and imaginary parts, so we can split a quaternion \textbf{q} into a \textit{scalar} part $S\qq = a$ and a \textit{vector} part $V\qq = b\qi + c\qj + d\qk$.
We can also define the \textit{conjugate} of a quaternion as $$ \qq^* = S\qq - V\qq = a - b\qi - c\qj - d\qk,$$ and the \textit{absolute value} of \qq$ $ as $$ \abs{\qq} = \sqrt{a^2 + b^2 + c^2 + d^2}.$$


\subsection{Geometry \& Quaternions}

\subsection{Differential Calculus \& Quaternions}
