Nobody knows how long vectors have been used in mathematics.
Some speculate that the parallelogram method for addition of vectors was lost in a work of Aristotle.
However quaternions, often defined as the quotient of two vectors, were not described until 1843 by William Rowan Hamilton.
Quaternions are now extensively used in aeronautics and computer graphics for their advantages over tradional transformtion methods.


\subsection{History of Quaternions}

On October 17, 1843, William Rowan Hamilton wrote a letter to his friend John t. Graves, Esq. on the subject of ``a very curious train of mathematical speculation."
The letter details his ``theory of quaternions", and follows his mathematical reasoning behind the development of his ``quaternions."
The day before, while walking across the Royal Canal in Dublin, Hamilton had the idea for the formula of quaternions as shown below in Equation \ref{eq:quat}.
\begin{equation}
\label{eq:quat}
i^2 = j^2 = k^2 = ijk = -1
\end{equation}
This equations and its implications will be investigated in Section \ref{sec:disc}.
The letter was only a few pages long, but it quite thorough in its' scope, and was eventually publiished in the \textit{London, Edinburgh, and Dublin Philosophical Magazine and Journal of Science} the following year in 1844.
Shortly after Hamilton's death in 1865, his son edited and published the longest of Hamilton's books, at over 800 pages.
Titled \textit{Elements of Quaternions}, this book was the go-to book on quaternions for several decades.
The wake of Hamilton's exploration into quaternions led to research associations like the Quaternion Society who described themselves as an ``International Association for Promoting the Study of Quaternions."

\subsection{Basic Geometric Transformations}

The primary topic of this manuscript is the application of quaternions to transformations in 3D space, so some terms will be defined here.
There are two methods of transformation: translation and rotation \cite{animation}.

\begin{defn}[Translation]
A \textit{translation} is a point in space moved from one position to another.
Let a point $P \in \mathbb{R}^3$ be denoted as $(x,y,z), x,y,x \in \mathbb{R}$ and the translation by a vector $(\Delta x, \Delta y, \Delta z)$.
Then the new position $P^\prime$ is $(x + \Delta x, y + \Delta y, z + \Delta z)$.
There is only one translation vector that takes $P$ to $P^\prime$.
\end{defn}

\noindent A \textit{rotation} can be defined in multiple ways.
The following definition is given by Euler, and will be used here.

\begin{defn}[Euler's Definition of Rotation]
Let $O, O^\prime \in \mathbb{R}^3$ be two orientations.
Then there exists an axis $l \in \mathbb{R}^3$ and an angle of rotation $\Theta \in [-\pi, \pi]$ such that $O$ yields $O^\prime$ when rotated $\Theta$ about $l$.
\end{defn}

\noindent It is important to distinguish between \textit{orientation} and \textit{rotation}.
Orientation here is the normal vector to an object in 3D space.
A rotation comprised of an axis and angle of rotation.
Unlike the translation between two points, the rotation between two orientations in 3D space is not unique.
\\ \\
\noindent Section \ref{sec:disc} will define and discuss quaternions in detail.
Section \ref{sec:comp} will compare quaternions and alternate methods of expressing rotations, while Section \ref{sec:app} will discuss real world applications.
Section \ref{sec:conc} will conclude the report. 
