This section will investigate alternative methods for rotating vectors in 3D, and compare their advantages and disadvantages to quaternions.

\subsection{Other Non-Euclidean Transformation Methods}
There are two alternate primary methods for expressing rotations of vectors in 3D: Euler angles and rotation matrices.
Euler angles, originally developed by Euler as a tool for solving differential equations, have become a widely used method for expressing rotation.
Euler angles ($\theta_x, \theta_y, \theta_z$) are the rotation angles about the x,y, and z axes respectively.
There are a few problems with Euler angles that mostly result from ambiguity.
The main problem is that there are multiple methods of using Euler angles to achieve the same rotation.
Take for example, a rotation about the z-axis by $180^\circ$.
Clearly, the simplest way to express this is with the Euler angles ($\theta_x = 0^\circ, \theta_y = 0^\circ, \theta_z = 180^\circ$).
However, another equally valid solution is ($\theta_x = 180^\circ, \theta_y = 180^\circ, \theta_z = 0^\circ$).
It is important to note that the \textit{order} that the Euler angles are applied makes a difference.
In the previous case it happened to not make a differnce, but in general, the angles are not commutative.
There are various conventions in place to help with the ambiguity.
In this paper, we will use the order of x,y, and then z.
\\ \\ The other primary method for expressing rotation in 3D is a rotation matrix.
A rotation matrix is usually method for applying Euler angles.
Each rotation about the x,y, and z axes are used to generate a 3x3 rotation matrix.
These are then multiplied together to get a single rotation matrix.
Sometimes a 4x4 homogenous matrix is used because it can hold translation and rotation in the same matrix.
We know that matrix multiplication does not commute, which is also true of rotations.
The following rotation matrices are used to describe rotation about the given axis:

\\ $$ R_x(\theta) =
\begin{bmatrix}
1	&	0 				& 	0 \\
0 	& 	cos(\theta)	&	-sin(\theta) \\
0 	& 	sin(\theta) 	& 	cos(\theta)
\end{bmatrix},
R_y(\theta) =
\begin{bmatrix}
cos(\theta)		&		0 		& 		sin(\theta) \\
0 				&		1		& 		0 \\
-sin(\theta) 	& 		0 		& 		cos(\theta)

\end{bmatrix},
R_z(\theta) =
\begin{bmatrix}
cos(\theta)		&		sin(\theta) & 		0 \\
-sin(\theta)	&		cos(\theta)	& 		0 \\
0 				& 		0 			& 		1
\end{bmatrix}
$$

So a rotation about the z-axis by $180^\circ$ would be represented by multiplying the three individual rotation matrices together.
\\
$$
\begin{bmatrix}
1	&	0 	& 	0 \\
0 	& 	1	&	0 \\
0 	& 	0 	& 	1
\end{bmatrix}
*
\begin{bmatrix}
1	&	0 	& 	0 \\
0 	& 	1	&	0 \\
0 	& 	0 	& 	1
\end{bmatrix}
*
\begin{bmatrix}
-1	&	0 	& 	0 \\
0 	& 	-1	&	0 \\
0 	& 	0 	& 	1
\end{bmatrix}
=
\begin{bmatrix}
-1	&	0 	& 	0 \\
0 	& 	-1	&	0 \\
0 	& 	0 	& 	1
\end{bmatrix} $$

Notice that this is the same final rotation matrix as if we had instead rotated the x-axis by $180^\circ$ and then the y-axis by $180^\circ$.
This can be verified as shown:

$$
\begin{bmatrix}
1	&	0 	& 	0 \\
0 	& 	-1	&	0 \\
0 	& 	0 	& 	-1
\end{bmatrix}
*
\begin{bmatrix}
-1	&	0 	& 	0 \\
0 	& 	1	&	0 \\
0 	& 	0 	& 	-1
\end{bmatrix}
*
\begin{bmatrix}
1	&	0 	& 	0 \\
0 	& 	1	&	0 \\
0 	& 	0 	& 	1
\end{bmatrix}
=
\begin{bmatrix}
-1	&	0 	& 	0 \\
0 	& 	-1	&	0 \\
0 	& 	0 	& 	1
\end{bmatrix} $$

\subsection{Comparison Between Methods}
Euler angles and rotation matrices have historically been used to represent rotation in 3D, and the mathematics behind them are well known.
Euler angles and rotation matrices make rotations about the x,y, or z axes simple.
However an arbitrary rotation about an arbitrary axis makes it difficult to derive the Euler angles for.
Another disadvantage, as briefly discussed above, is that the order that the rotation matrices are applied in matter.
Different conventions lead to different results so care must be taken when coding and using libraries.
Another disadvantage is gimbal lock, which will be investiagated in more depth in the Applications section.
In addition, given a rotation matrix, it can be difficult to find an inverse.
Lastly, the homogenous matrix holds extra information.
When coding applications, extra information is a waste of program space.
\\ \\ \noindent The biggest advantage to Euler angles and the corresponding rotation matrices, other than for historical reasons, is the ability to encode translation, rotation, scaling, and projection into one matrix.
This can be useful 
